% Issue 1

%   packages:
\documentclass{article}
\usepackage[utf8]{inputenc}
\usepackage[T1]{fontenc}
\usepackage{microtype}
\usepackage{graphicx}
\usepackage{fancyhdr}
\usepackage{float}
\usepackage{lipsum}
\usepackage{newspaper}
\usepackage{newspaper-mod}
\usepackage{multicol}
\usepackage{picinpar}

% uasage of picinpar:
% \begin{window}[1,l,\includegraphics{},caption]xxxxx\end{window}

% These can be issued just before any
% byline or headline in the paper, to
% individually style each article
% \renewcommand{\headlinestyle}{\itshape\Large\lsstyle}
% \renewcommand{\bylinestyle}{\bfseries\Large\raggedright}


% TODO: Change the font to Times New Roman

% \setlength{\parindent}{2pt}
%To get rid of indents 


%Metadata you shouldn't change
\SetPaperName{
          %  \includegraphics[scale=0.3]{fps}
        % \scalerel*{\includegraphics{fps}}{f}
            FPS Eagles:   
         \hfill  \includegraphics[scale=0.1]{mascot} 
      }
\SetHeaderName{FPS Eagles}
\SetPaperLocation{
%    \centering
   %\begin{figure}
    %   \centering
  %  \vspace{5mm}
  %  \includegraphics[scale=0.12]{lfps} 
   %  \includegraphics[scale=0.09]{W}
   %\end{figure}
    
   
}
\SetPaperSlogan{
            %     \includegraphics[scale=0.3]{fps}
        %        FIRST ISSUE!
                
                ``\textit{FPSonians are the best; second to none!}''
              %  \includegraphics[scale=0.1]{cfps}
                }
\SetPaperPrice{By Awab Qureshi and Sukaina Kazmi; batch 2020}
%\cfoot{\textit{First issue ever! FPSonians are the best; second to none!}}



% Metadata you should change

\date{February 2020}
\currentvolume{1}
\currentissue{1} 

\begin{document}
\maketitle
\large 

%                               1ST PAGE

\headline{PLASTIC}\textbf{by Awab Ahmed Qureshi XI-D}

Researchers recently probed fish-life in the deepest parts of our ocean and found something shocking. In their digestive tracts were quantities of micro-plastics. The study raised several concerns. Finding micro-plastics at such a depth means that even the deepest parts of our oceans are not safe from our litter.More and more plastic is thrown into our oceans every day with some estimates ranging between 4.8~12.7 million tonnes per Annum.\newline Indeed micro-plastics are quite dangerous as the ones found in such fish could easily transfer up the food chain and into our food supply, causing a plethora of adverse health effects.

Students in the SLC in FPS realised this crisis, and thus, to prevent it, worked with the admin to ban the sale of all single-use plastic bottles in the canteen. The ban, which was put in place earlier this term has reduced the plastic bottle waste OLDC generates to $\boldsymbol{0}$. In place of this, students can now refill their bottles at a lower rate from the canteen's water filter.

It is hoped that our contribution, whilst small, will make a difference by discouraging the use of disposable plastics and will be a step on the long road towards a green future. 



%do \closearticle for that double line bar

\begin{multicols}{2}
%to make columns

\includegraphics[scale=0.2]{w}

\includegraphics[scale=0.45]{noplastic}

\includegraphics[scale=0.1]{w}

%\headline{Escape to Wonderland}\textbf{by Sukaina Kazmi XI-C}

%Friday 11th October was a night to remember for all grade 11s. FPS A-Levels had held its 4th annual “Escape to Wonderland” which unfortunately the last of its kind. The activity aided in enhancing and bringing out a student’s team building and problem solving skills as the students hurriedly pieced together multiple clues and puzzles to save the world from the apocalypse .Students in teams of eight were locked in a military bunker with an attendant and had approximately an hour to stop the virus from spreading into the atmosphere and escaping the room. Students had to channel their cognitive skills to scavenge the keys to the buzzer thereby emerging triumphant. Once the students succeeded in escaping the room they gained entry to the magical wonderland full of food, games, rides and music for them to dance the night away. The night ended up being a truly memorable event which I’m sure this batch is to remember.



\headline{Happiness fest}\textbf{by Laiba Abid XI-C}


To mark the end of the term, and to help students unwind from the tiring exams, the SLC and school organised The Happiness Fest after the paper review. The event featured numerous activities for students to take part in, such as karaoke, and a swinging ship. Music played throughout the event and pupils could request and dedicate any song to anyone of their choosing. The KCC stall sold a variety of accessories to raise money for charity. Along with all this there was also a food street in the parking lot. Several stalls were present offering many food options to appease everyone’s appetites. The day culminated in a performance by our talented young musicians, followed by an inter-house football tournament. 

Overall it was a fun-filled event, and a great opportunity to relax with friends.



\end{multicols}
\pagebreak 

%to make sure stuff stays on this page

%                               2ND PAGE 

\headline{Editorial: Our Lack of Awareness}\textbf{by Arham Siddiqui XI-B}

The lack of awareness on gender rights, hygiene, citizen rights etc. is unfortunately an issue Pakistan struggles with. Even worse is the realization that millions of people in rural and urban settings are not aware of how they should be treated by the people around them, by their rightful authorities or even how they should treat their own selves. 

Whilst being a fact more commonly widespread nowadays, female oppression in workplaces and in their personal lives is still a big issue. Millions of Pakistani women forcefully succumb to fulfilling their stigmatised role in society instead of their aspirations because they are not educated on their rights as a woman. This lack of awareness creates more problems for females in different walks of life.

Another instance where lack of awareness hinders Pakistani society is in hygiene, which has caused a highly escalating infant mortality rate. This problem is further exacerbated in rural areas of the country, where sanitary methods of keeping newborns clean and healthy during their weaning phase is a null prospect. No proper post-natal care is propagated, which results in thousands of kids failing to reach the age of 5. 

The people of Pakistan have also never been made fully aware of the democratic rights they possess, which has led to Pakistan wading in turbulent waters. Many citizens far and wide remain complacent, not willing to put in the effort of exercising their voice by casting a vote. When faced with sights of social injustice the people shrug it off as something that cannot be changed. This has not only made the common man more uninterested in local affairs, but has also led to the country sinking into a state of regression. It is no secret that democracies fail when citizens become disinterested. 

A viable approach to all this is education. Together, as a single person or neighbourhood - with or without the government - we can spread awareness of our rights. Fortunately change has already begun. As the social movements in recent years have shown, the young populace is becoming acutely aware of their rights. May this create a better society for us all.


\closearticle

\begin{multicols}{2}
\headline{Sports: Sarah Sajjad}


Sarah Sajjad is an ambitious champion who represented us in the National Olympic Games held in Peshawar, and has impressed us once again as the youngest throw-ball athlete of the event. This is not the first time Sara has made FPS proud as the outstanding athletic virtuoso also made us all proud last year in inter-school competitions. 

She left for Peshawar, along with her mother, on Nov. 8th. The team started their first match with Balochistan, winning it with a lead of 20 points. Among the 6 games that were held: two qualifying matches, two semi-finals and two finals; Sindh won all 6 of them. Sara honored us all, as well as her family, and most importantly her parents, with her trophy and gold medal.  




\begin{center}
    \includegraphics[scale=0.4]{athlete}    
\end{center}


%\headline{This Week in History}1912-The ocean liner Titanic sinks after hitting an iceberg in the North Atlantic Sea.1966-Football’s World Cup is stolen from a display in London. A dog finds it a week later, under a hedge.1968-The funeral of Martin Luther King Jr. takes place.1973-The art world mourns Spanish artist Pablo Picasso who dies in France aged 91.




\end{multicols}

\closearticle

\pagebreak


%                               3RD PAGE

\headline{What's Up With Tech?}
\begin{multicols}{2}
    \headline{Xbox series X}
We have been waiting to find out about the xbox replacement for a while now, but details have been kept under wraps. That is, until now, as Microsoft has released not only details, but even pictures.

The new console will include a high-speed solid-state hard drive (SSD) to lessen game load times, and a powerful new chip-set to make your games look even better. Its polarising shape has also generated buzz, with people remarking how it resembles a PC more than a console.

Microsoft is expected to release the "Xbox series X" late in the year for around £600.
%\closearticle


\headline{Half-Life: Alyx}
Valve has announced a new Half-Life game in over a decade. The news shocked fans, who thought they'd never see another instalment in the series again. Many memes were popularised as fans grew restless such as the famous "Half-Life 3 confirmed" and "Valve can't count to 3".

The game will be entirely in Virtual Reality, causing many to hope that it will bring VR headsets to general consumer's homes and pave the way for a new era. 

Half-Life: Alyx releases in March and is likely to generate more hype and possibly break the internet at release.

%\vfill
%\closearticle

\end{multicols}
\closearticle

\begin{center}
\headline{Jokes, and Memes}
\includegraphics[height=5cm]{m1.jpeg}
\includegraphics[height=5cm]{m4.jpeg}

\includegraphics[height=6cm]{m2.jpeg}
% \includegraphics[height=5.5cm]{m3.jpeg}

\end{center}

\pagebreak


%                               4TH PAGE

\begin{multicols}{2}

\headline{What about the Grammy's?}

The Grammy's are the music industry's most prestigious award and in the past have been won by greats such as Daft Punk, Taylor Swift, Adèle, Bruno Mars, and others. The award's iconic shape is that of a grammophone playing a vinyl record and is presented in many different categories such as best album, song, new artist and lifetime achievement among others.

This year, the Grammy's faced another controversy as ten days before the awards the Recording Academy placed Dugan on administrative leave, accusing her of vaguely defined “misconduct” against a senior female employee, which seemingly amounted to some possible verbal abuse. Four days later, Dugan’s attorneys hit back with her bombshell complaint to the EEOC.

But the awards still went on and on Jan. 26, Billie Eilish's "Bad Guy" became the 11th consecutive winner for record of the year to have been a top five hit on the Billboard Hot 100. The feat establishes a new record streak, surpassing the 10 consecutive record of the year champs that doubled as top five hits from Eagles' "Hotel California" in 1977 through Steve Winwood's "Higher Love" in 1986.


\vfill
\includegraphics[scale=0.1]{w}

\includegraphics[scale=0.35]{grammy}

\includegraphics[scale=0.1]{w}
\vfill

\headline{Prodigy?}

\includegraphics[scale=1]{prodigy}


Boy to graduate aged nine-and his university says he’s ‘three times smarter’ than anyone they’ve ever taught. Laurent Simon still finds time to play Fortnite and Minecraft, and to look after his young puppy named Sammy. This nine year old boy, inspired by Nikola Tesla, is preparing to become the youngest ever graduate. He is studying electrical engineering at University of Technology in Eindhoven in the Netherlands, heading down the home-straight of a degree that has taken him just nine months complete. It would normally take them three years to graduate but Laurent, with an IQ of 145, will do so next month. The boy wonder says he is looking forward to taking a holiday as the Christmas season beckons, but also already has plans to start a PhD and "study a little medicine".He says he wants to research artificial organs and develop a Terminator-style artificial body, with his Instagram page sporting an image of Arnold Schwarzenegger posing alongside a model of his iconic T-800 skeleton. As for what he has been getting up to in Eindhoven, he has been working on a research project involving "placing neurons and making connections" to test the effects of different medication on the brain. If that sounds beyond the realms of possibility for a boy not yet old enough to watch the latest Spider-Man film in a UK cinema by himself, it might be comforting to know that he still finds time for more simple childhood pleasures.

\pagebreak


\end{multicols}


\end{document}